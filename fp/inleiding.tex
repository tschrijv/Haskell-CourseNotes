
This chapter discusses several practicalities related to the Haskell course
and provides a background on Functional Programming in general and on Haskell
specifically.

Before we dive into practical programming with the functional programming language
Haskell, we briefly consider the mathematical foundation of (functional) programming
languages. Even though a number of concepts may seem initially rather
abstract, we will gradually discover that they provide the underlying philosophy
around which Haskell is built. Without insight into this philosophy, we cannot
understand the essence of functional programming.

We also briefly summarize
the history and impact of the Haskell language, and mention a number of
useful Haskell-related sources and tools.

%===============================================================================
\section{Course Setup}

The Haskell course which this book is a part of is organised around
practicing---the only way to learn programming---uses a flipped classroom
teaching style for that.
\begin{description}
\item[Before the lecture:]
You prepare for the lecture by reading---in
advance---a chapter of this book and by making some basic preparatory exercises. 
\item[During the lecture:]
We focus on getting a deeper understanding of the material by addressing any questions and difficulties you have putting what you have read into practice by solving (basic and advanced) exercises together, discussing alternative solution strategies, \ldots
\item[After the lecture:]
You get additional practice by making additional exercises more indenpently.
\end{description}

To help you along we have a whole support system in place:
\begin{description}
\item[The e-systant platform:]
We have created the \url{http:://esystant.be} exercise platform for this
course. The platform allows you to make Haskell exercises in the browser
without having to install anything. E-systant contains all the course's
exercise assignments. It gives automatic feedback on your solutions, allows
you to improve them and also provides model solutions. You can keep track
of your progress relative to your peers on the leaderboard.
You can use e-systant at the exam; it includes all the electronic documents that you can consult at that time.

\item[Slack workspace:]
Since the corona crisis we are using a slack workspace both during and
outside of the lectures. Students loved this and strongly recommended that we
keep this for you.

We use the workspace during and outside the lectures (it's there 24/7) to
exchange code snippets, ask questions, give clarifications, discuss, vote,
post memes, \ldots

\item[Coaches:]
Some courses have teaching assistants; we have coaches. They are there
to help you become better Haskell programmers. Be sure to talk to them
regularly and discuss code you have written. Whether you are stuck or
think it is already perfect, our coaches will help you go further.

\end{description}

%===============================================================================
\section{Beyond the Course}

This course only covers the basics of Haskell and functional programming. For
students who have a taste for more, I recommend reading the Advanced Topics
part of this book and consider one or more follow-up activities:
\begin{itemize}
% \item De studentenkring Zeus organiseert een tweemaandelijkse bijeenkomst voor
%       ge\"interesseerden in Haskell, genaamd GhentFPG.
\item The Capita Selecta: Artificial Intelligence course consists
      of several tracks, one of which is ``Language Engineering in Haskell''.
\item You can conduct further research into Haskell and other (functional)
      programming languages during a master thesis, and possibly during a subsequent PhD.
\item Utrecht University organises a yearly summer school
      where you can expand your practical experience with Haskell.
\item Haskell has a very active open source community
      that is welcoming to newcomers. The community participates every year
      in the Google Summer of Code. Just like former students of this book
      you can participate.
\item There are interesting international internship opportunities. 
      For instance, one former student of this book has done an internship
      at Tsuru Capital in Tokyo, a second 
      at Google in Zurich and a third at CentralApp in Brussels.
\item There are interesting career opportunities using Haskell. 
      Former students of this course work at CentralApp, Fugue, PieSync, Standard Chartered, and Tweag.
\end{itemize}

Note that knowledge of Haskell is also an asset for companies that do not
make use of functional programming, but are knowledgeable about the domain.
It is often used as a filter to identify the better programmer; of course
everyone knows Java, but Haskell is a different matter.

%===============================================================================
\section{The $\lambda$-calculus}

Functional programming goes back to the time immediately before the development
of the first computer and the advent of programming languages. The first
computer scientists, who were mathematicians, were already preparing their arrival.
They were studying at a fairly abstract level what could and could not be computed
with a computer. For this purpose \textbf{Alonzo Church} developed in 1936 
the first version of the $\lambda$-calculus. 

This $\lambda$-calculus captures the essence of a programming language, but
is on the one hand too \emph{minimalistic} and on the other hand too \emph{abstract}
to be able to speak of a proper programming language.

The minimalism means that this calculus has just enough features to study
computability without needlessly complicating the mathematical study of this
phenomenon. As a consequence, the calculus can \emph{in theory} express practical
computations, but is \emph{in practice} too impractical to do this effectively. 
Of course we cannot condemn the calculus for that, as it was never
its intended use. Concretely the calculus consists of only 3 different expression
forms and 3 different calculation rules.

%-------------------------------------------------------------------------------
\subsection{Expressions}

A program in the calculus is an \emph{expression} $E$. Such an expression 
can be one of three different possible forms.
\begin{framed}
\begin{equation*}
\begin{array}{rcl@{\hspace{3cm}}l}
E & ::=  & x                    & \text{variables}\\
  & \mid & \lambda x.E          & \text{(function) abstractions} \\
  & \mid & E_1\,E_2             & \text{(function) applications}
\end{array}
\end{equation*}
\end{framed}

The central form\footnote{It is this form that gives functional
programming its name.} is the function abstraction $\lambda x.E$, which is
also called 
$\lambda$-abstraction, abstraction or simply function. This
$\lambda$-abstraction represents a function in 1 parameter $x$ and a function body
$E$ that may refer to the parameter $x$.  

A parameter $x$ presents a variable that can be used in the body of a function.
An example of a very simple function is $\lambda x.x$,
which is the function that just returns its parameter. This function
is known as the \emph{identity function}.

The third expression form is the function application, or application for short,
$E_1\,E_2$. This form applies a function expression $E_1$ to another expression
$E_2$. An example of such an application is the application of the identity function
to the variable $y$: $(\lambda x.x)\,y$.

The $\lambda x$ part of the notation is called the
\emph{$\lambda$-binder}, because it binds the parameter $x$ to a value during function application.
Within an expression $E$ we distinguish the \emph{bound variables} from the \emph{free variables}.
A bound variable is one that occurs in $E$ ``under'' a binder that binds it.
A free variable is one that is not bound by a binder. For example,
in $\lambda x.(\lambda y.(x\,y))\,z$ both
$x$ and $y$ are bound variables, while $z$ is free. 

Sometimes there can be multiple binds for the same variables. For example,
there are two binders for $x$ in $\lambda x.(\lambda x.x)$. In this case,
every occurrence of a variable is bound by its nearest (i.e., innermost) binder.
We say that the inner binder shadows the outer binder.
If there are multiple occurrences of a variable, each occurrence may be bound by
a different binder. For example, in $\lambda x.((\lambda x.x)\,x)$ the $x$ on the left
is bound by the inner binder, while the $x$ on the right is bound by the outer binder.
The same variable can also occur both free and bound. For instance, in $(\lambda x.x)\,x$
the $x$ on the left is bound, while the one on the right is free.

%-------------------------------------------------------------------------------
\subsection{Calculation Rules}

On their own the expression forms are not very useful: we can put together
various expressions, but have not much to say about them at this point.
The expressions are only notation (\textit{syntax}) without meaning (\textit{semantics}).

The calculation rules assign a kind of meaning to the expressions. There are three
calculation rules for the $\lambda$-calculus, given in the form of equations:

\begin{framed}
\begin{equation*}
\begin{array}{rcl@{\hspace{3cm}}l}
 \lambda x.E & \equiv & \lambda y.[y/x]E           & \text{$\alpha$-conversion} \\ \\
(\lambda x.E_1)\,E_2 & \equiv & [E_2/x]E_1         & \text{$\beta$-reduction}   \\ \\
\lambda x.(E x) & \equiv & E         & \text{$\eta$-conversion}   
\end{array}
\end{equation*}
\end{framed}

\paragraph{$\alpha$-conversion}
You have probably already developed the following intuition in the context of
other programming languages you know: the names of parameters are irrelevant,
you can freely change them without affecting the behavior of your program.
This is also true in the $\lambda$-calculus. Whether you write the identity
function as $\lambda x.x$ or as $\lambda y.y$ makes no difference: the two
expressions are equivalent.

The $\alpha$-conversion rule expresses this formally. The rule
makes use of the generic notation
$[y/x]E$ for renaming variables. This notation expresses that
we apply the substitution to the expression
$E$. This means that we replace in $E$ every \textit{free} occurrence of $x$ 
with $y$. For example, $[y/x]x$ results in $y$, but $[y/x](\lambda x.x)$ remains $\lambda x.x$
because $x$ does not occur free.

While reasoning about the equivalence of two expressions, the
$\alpha$-conversion is used to bridge the trivial differences in variable
naming. A second important application of $\alpha$-conversion is to ensure that
every $\lambda$-binder in an expression $E$ binds a variable with a different name.
This avoids confusion and unintended shadowing.

\paragraph{$\beta$-reduction}
The rule for $\beta$-reduction is closely connected to the evaluation
of programs in an actual programming language. It allows us to \emph{evaluate}
an expression by simplifying or \emph{reducing} it. The rule replaces a function
application $(\lambda x.E_1)\,E_2$ by the function body
$E_1$ in which the parameter $x$ has been replaced by the argument
$E_2$. Just like the previous rule, this rule expresses the replacement in terms of substitution.
This means that only the free occurrences of the parameter $x$ are replaced.

\paragraph{$\eta$-conversion}
The 
$\eta$-conversion rule is an additional rule for comparing expressions without
evaluating them. The rule expresses the notion of \emph{extensionality}:
the two expressions $\lambda x.(E\,x)$
(where $x$ does not occur freely in $E$) and $E$
behave in the same way modulo $\beta$-reduction.
If we  $\beta$-reduce $(((\lambda x.)E\,x)\,E')$,
for any $E'$, we get $E\,E'$. Hence, modulo $\beta$-reduction
we cannot distinguish the two expressions. This fact is captured 
in the rule for $\eta$-conversion.

This rule is often applied to write expressions more compactly.

\paragraph{Equivalence}
The three calculation rules are expressed as axioms of the equivalence relation
$\equiv$. Three additional, usually implicitly assumed, defining axioms of an equivalence relation
are 
\textit{reflexivity}, \textit{symmetry} and \textit{transitivity}. 
A final special axiom is \emph{Leibniz's law}. This law expresses that
we may replace a subexpression 
$E_1$ of a larger expression $E$ by an equivalent subexpression $E_2$ and still preserve
equivalence of the whole.
\begin{framed}
\begin{equation*}
\begin{array}{rcl@{\hspace{1cm}}l}
 E & \equiv  & E                    & \text{reflexivity} \\ \ \\[0cm] 
 E_2 & \equiv & E_1\hspace{1cm}\text{if }E_1 \equiv E_2 & \text{symmetry} \\ \ \\[0cm]
 E_1 & \equiv & E_3\hspace{1cm}\text{if }E_1 \equiv E_2\text{ en }E_2 \equiv E_3 & \text{transitivity} \\ \ \\[0cm]
 [{E_1}/x] E  & \equiv & [E_2/x]E  \hspace{1cm}\text{if }E_1 \equiv E_2     & \text{Leibniz' law}
\end{array}
\end{equation*}
\end{framed}

\paragraph{Variable Capture}
A problematic situation may arise during $\beta$-reduction that
inadvertently changes the meaning of an expression.
Consider the expression
$(\lambda y.(\lambda x.(x\,y)))\,x$. If we carelessly apply $\beta$-reduction
to the outermost application, we obtain $\lambda x.(x\,x)$. Yet, this renders
the second occurrence of $x$, which was originally free, into a bound variable.
We say that the variable has been captured by the $\lambda$-binder. This does
not happen when we start from an $\alpha$-renamed variant of the original
expression
$(\lambda y.(\lambda z.(z\,y)))\,x$, which reduces to
$\lambda z.(z\,x)$. 
We consider the first variable-capturing $\beta$-reduction to be invalid.
In practice, we always make sure to first rename bound variables to avoid
variable capture from happening.

%-------------------------------------------------------------------------------
% \subsection{Naar volwaardige programmeertalen}

%===============================================================================
\section{Haskell}

This book teaches functional programming in Haskell. Haskell is the mainstream
functional programming language that most closely embodies the principles of
functional programming.  That is why we call it a \emph{pure} functional
language; it is not diluted by concessions to the imperative paradigm of
computer hardware and mainstream programming languages. An additional feature
of Haskell is its \emph{laziness}, which we discuss at length in a later
chapter.

%-------------------------------------------------------------------------------
\subsection{History}

The '80s were abuzz with research into lazy functional programming languages.
Many researchers built their own experimental language implementation, of which 
the commercial Miranda system (1985) is by far the most well-known.
At the end of the '80s many researchers decided that it made more sense to join
forces than to continue dabbling with their individual systems. They founded the
\emph{Haskell committee} to design a common lazy functional programming language.

\begin{sidenote}
\paragraph{What's in a name}
The name Haskell has been derived from the American logician
\emph{Haskell B. Curry}
(1900-1982). He studied the essence of programming by means of minimal calculi known as
\emph{combinators}. We will encounter his name twice more, when we discuss the 
\emph{Curry-Howard isomorphism} and the concept of \emph{currying}.\footnotemark[1] Haskell Curry has also given his name to another pure functional language,
Curry, which contains elements of logic programming.
\ \footnoterule \ \\
{\footnotesize\footnotemark[1]{The name currying is actually inappropriate as the concept had been invented previously by Sch\"onfinkel.}}
\end{sidenote}

The Haskell committee released Haskell 1.0 in 1990.
In subsequent years new versions of the language were released until
finally a stable standard was published in 1998: Haskell'98. At that point
the committee was effectively disbanded, but the development of different
Haskell systems did of course continue. Recently the Haskell committee 
was re-instituted by the Haskell' (pronounced ``Haskell \emph{Prime}'') movement to release updates to the
Haskell standard on a yearly basis.

%-------------------------------------------------------------------------------
\subsection{Prominent Language Properties}

There are many interesting aspects to the Haskell language and its ecosystem.

\paragraph{Types}
Haskell is known for its sophisticated type system. The language itself
is \emph{strongly and statically} typed, and really puts the types to
work for the programmer instead of the opposite.

Strong typing means that every value in the program has a unique associated type
and that only operations that conform to that type can be applied to it. There 
is no wiggle room.

Static typing means that the types and possible type errors are established
during compilation, and thus before execution. In contrast to most statically
typed languages, the programmer does not have to declare much type information.
The Haskell compiler can mostly determine the types on its own using a process
known as \emph{type inference}.

\paragraph{Abstraction}
Haskell has a large capacity for abstraction. Many repetitive code patterns can
be isolated in named entities and reused under that name.

The threshold for code reuse is a lot lower than with the typical unit of
abstraction in object-oriented languages, the class. Hence, a Haskell program
is often assembled from many small entities that are quickly composed into
various new combinations. Large screen-filling code blocks and copy-paste
programming are simply not done in Haskell.

As a consequence, Haskell does not need books about \emph{design patterns}, 
but instead provides libraries with reusable abstractions. Haskell programs
are thus small and make ample use of libraries.

\paragraph{Domain-Specific Languages}
The emphasis on small reusable abstractions, combined with advanced syntax capabilities,
make Haskell a good stomping ground for \emph{embedded domain-specific languages} (EDSLs).
The core idea is to offer in Haskell a combination of abstractions that are useful
for solving problems in a particular problem domain (e.g., drawing figures,
managing financial contracts or controlling robots). By composing these abstractions
in various ways new problems in the problem area can be solved quickly. Thanks to 
suitable syntax and cohesive abstractions it seems as if the programmer is no longer
working in Haskell, but in a custom domain-specific programming language.


\paragraph{Learning Curve}
The Haskell learning curve looks quite different from that of Java.
On the one hand it is much steeper. This means that the language requires 
more effort and insight of the programmer to reach a certain skill level. For instance,
the programmer has to be able to handle more abstract concepts, and identify their potential
application to concrete problems.
On the other hand Haskell provides a much richer learning experience. Haskell facilitates
continued learning about programming concepts and different ways to tackle
programming problems. Even if you do not end up programming in Haskell in the long run, many
of the ideas learnt in Haskell can be applied in the context of other languages.
Moreover, once mastered, Haskell is very powerful. For instance, Haskell programs are
typically $10\times$ shorter than the corresponding Java programs.

%-------------------------------------------------------------------------------
\subsection{Industrial Impact}

The general presence and visibility of Haskell in the software industry is
rather limited. Nevertheless, the language has a substantial indirect influence
on mainstream programming. Many ideas and concepts are first conceived, developed 
and fine-tuned in the context of Haskell before the most successful ones trickle
down and are adopted by the mainstream. A good example is Scala, which was
mainly influenced by Haskell, and which in turn influences Java and C\#.
In general it is fair to say that the development of mainstream programming
languages is very sluggish, and that Haskell is one of the trailblazers
far ahead of the pack.

Due to its steep learning curve Haskell is less suitable for (large) companies
with a lower-educated programmer population. Instead, Haskell can often be
found in small, highly productive programmer teams (inside or outside of large
companies), such as at AT\&T, Bluespec, Facebook, Galois, Linspire, Qualcomm
and Google. In addition, Haskell also thrives in application niches with
highly educated programmers.  The most prevalent such area is that of
\emph{investment banks}: ABN AMRO, Barclays Capital, Credit Suisse, Deutsche
Bank, Standard Chartered and Tsuru Capital.


%-------------------------------------------------------------------------------
% \subsection{Practicalities}
% 
% \paragraph{Haskell System}
% This book uses the Haskell Platform\footnote{\url{http://hackage.haskell.org/platform/}}
% which simultaneously installs the famous Glasgow Haskell Compiler (GHC), the most important libraries
% and the \texttt{cabal} package manager. Do not proceed before installing this platform. 
% % Wie niet over een eigen computer beschikt kan terecht
% % op Helios\footnote{\url{http://helpdesk.ugent.be/helios/}} waar GHC reeds
% % beschikbaar is.
% 
% \paragraph{Documentation}
% There are many online information sources that teach various aspects of Haskell.
% A general starting point is \url{http://haskell.org}, the
% official Haskell website. Haskell-related discussion forums are the official
% mailing list
% \url{haskell-cafe@haskell.org} and subfora on well-known sites like
% Reddit and StackOverflow. More information and documentation on GHC can be found
% at \url{https://www.haskell.org/ghc/}. 
% % Voor specifieke vragen over
% % deze cursus kan je in eerste plaats terecht bij het Minerva-forum en bij de
% % docent.
%  
% 
